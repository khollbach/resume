% (c) 2002 Matthew Boedicker <mboedick@mboedick.org> (original author) http://mboedick.org
% (c) 2003-2007 David J. Grant <davidgrant-at-gmail.com> http://www.davidgrant.ca
% (c) 2008 Nathaniel Johnston <nathaniel@nathanieljohnston.com> http://www.nathanieljohnston.com
% (c) 2011 Scott Clark <sc932@cornell.edu> http://cam.cornell.edu/~sc932
% (c) 2024 Kevan Hollbach <khollbach@gmail.com>
%
% This work is licensed under the Creative Commons Attribution-Noncommercial-Share Alike 2.5 License. To view a copy of this license, visit http://creativecommons.org/licenses/by-nc-sa/2.5/ or send a letter to Creative Commons, 543 Howard Street, 5th Floor, San Francisco, California, 94105, USA.

\documentclass[letterpaper,11pt]{article}
\newlength{\outerbordwidth}
\pagestyle{empty}
\raggedbottom
\raggedright
\usepackage[svgnames]{xcolor}
\usepackage{framed}
\usepackage{tocloft}
\usepackage{etoolbox}
\robustify\cftdotfill
\usepackage{verbatim} % for comment environment
\usepackage{hyperref}
\hypersetup{
    colorlinks=true,
    urlcolor=blue,
}
\urlstyle{same}

%-----------------------------------------------------------
%Edit these values as you see fit
\setlength{\outerbordwidth}{3pt}  % Width of border outside of title bars
\definecolor{shadecolor}{gray}{0.75}  % Outer background color of title bars
                                      % (0 = black, 1 = white)
\definecolor{shadecolorB}{gray}{0.93}  % Inner background color of title bars

%-----------------------------------------------------------
%Margin setup
\setlength{\evensidemargin}{-0.25in}
\setlength{\headheight}{-0.25in}
\setlength{\headsep}{0in}
\setlength{\oddsidemargin}{-0.25in}
\setlength{\paperheight}{11in}
\setlength{\paperwidth}{8.5in}
\setlength{\tabcolsep}{0in}
\setlength{\textheight}{9.75in}
\setlength{\textwidth}{7in}
\setlength{\topmargin}{-0.3in}
\setlength{\topskip}{0in}
\setlength{\voffset}{0.1in}

%-----------------------------------------------------------
%Custom commands
\newcommand{\resitem}[1]{\item #1 \vspace{-2pt}}
\newcommand{\resheading}[1]{\vspace{0pt}
  \parbox{\textwidth}{\setlength{\FrameSep}{\outerbordwidth}
    \begin{shaded}

\setlength{\fboxsep}{0pt}\framebox[\textwidth][l]{\setlength{\fboxsep}{4pt}%
\fcolorbox{shadecolorB}{shadecolorB}{\textbf{\sffamily{\mbox{~}\makebox%
[6.762in][l]{\large #1} \vphantom{p\^{E}}}}}}
    \end{shaded}
  }\vspace{-5pt}
}

\newcommand{\ressubheading}[4]{
\begin{tabular*}{6.5in}{l@{\cftdotfill{\cftsecdotsep}\extracolsep{\fill}}r}
		\textbf{#1} & #2 \\
		\textit{#3} & \textit{#4} \\
\end{tabular*}\vspace{-6pt}}

%-----------------------------------------------------------
\begin{document}

\textbf{\Large Kevan Hollbach} \hfill {\sffamily \href{https://www.linkedin.com/in/kevan-hollbach/}{linkedin.com/in/kevan-hollbach}}

{\sffamily khollbach@gmail.com} \hfill {\sffamily \href{https://github.com/khollbach/}{github.com/khollbach}}

\resheading{Work Experience}

\textbf{Professional Sabbatical} \hfill 2023

Attended the Recurse Center, a computer programming retreat. Selected projects:
\begin{itemize}
\item {\sffamily \href{https://github.com/khollbach/apple-ii-demos/}{Demos}} and {\sffamily \href{https://github.com/khollbach/apple-ii-games/}{games}} written in assembly and C for the Apple II home computer \\[-2pt]
\item {\sffamily \href{https://github.com/khollbach/apple-ii-bas2wav/}{bas2wav}}: Rust tool to convert Applesoft BASIC source code into .wav audio \\[-2pt]
\item {\sffamily \href{https://github.com/khollbach/chip-8/}{CHIP-8}}: retro video game system emulator, written in Rust \\[-2pt]
\item {\sffamily \href{https://github.com/khollbach/space-invaders/}{Space Invaders}}: video game for an embedded system, written in Rust
\end{itemize}

\textbf{Software Engineer}, Google, San Francisco \hfill June 2022 -- January 2023
\begin{itemize}
\item Wrote CLI tools in Rust to manipulate the content of Fuchsia archive format files \\[-2pt]
\item Migrated C++ code for touchscreen gesture detection to next generation graphics/input stack
\end{itemize}

\textbf{Software Engineering Intern}, Cruise, San Francisco \hfill Spring 2020
\begin{itemize}
\item Developed machine learning infrastructure in Go for a self-driving car \\[-2pt]
\item Added lidar point cloud metrics to a production big data pipeline
\end{itemize}

\textbf{Software Engineering Intern}, Rubrik, Palo Alto \hfill Spring 2019
\begin{itemize}
\item Wrote production Scala code for a scalable, fault-tolerant distributed system \\[-2pt]
\item Improved VM snapshot recovery performance by shortening archived diff-chains
\end{itemize}

\textbf{Software Engineering Intern}, NVIDIA, Santa Clara \hfill Summer 2018
\begin{itemize}
\item Designed and wrote embedded C++ code for the Tegra SoC platform to control \\ clocks and resets, I$^2$C bus functionality, and hardware security module behaviour \\[-2pt]
\item Worked with and debugged QNX resource managers and Linux device drivers
\end{itemize}

\resheading{Education}

\textbf{Research-based Master of Science}, University of Toronto \hfill 2018 -- 2021

Theoretical Computer Science
\begin{itemize}
\item Research area: distributed systems, computability in the asynchronous shared memory model \\[-2pt]
\item Thesis: Every object type is equivalent to some linearizable object type
\end{itemize}

\textbf{Honours Bachelor of Science}, University of Toronto \hfill 2013 -- 2018

Computer Science (Focus in Computer Systems, Focus in Theory of Computation)

Mathematics Minor
\begin{itemize}
\item Grades of 95\% or higher in five graduate-level computer science courses: \\ Theory of Distributed Computing, Algorithm Design \& Analysis, Graph Theory, \\ Advanced Topics in Distributed Computing, Topics in the Theory of Computation
\item Grades of 90\% or higher in many fourth-year computer science courses; e.g., \\ Compilers \& Interpreters, Advanced Computer Networks, Complexity \& Computability
\end{itemize}

\end{document}
